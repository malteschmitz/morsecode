\documentclass{scrartcl}

\KOMAoptions{headings=normal,
             fontsize=9pt,
             parskip=half,
             headsepline,
             twoside,
             captions=rightbeside,
             captions=centeredbeside}

\usepackage[utf8]{inputenc}
\usepackage[ngerman]{babel}
\usepackage[T1]{fontenc}
\usepackage{lmodern}
\usepackage{textcomp}

\usepackage{geometry}
\geometry{paper=a5paper,
          inner=1cm,
          outer=1cm,
          top=1cm,
          bottom=1cm,
          includehead}

\usepackage[scaled=0.92]{helvet}
\renewcommand*\familydefault{\sfdefault}
\usepackage{courier}
\usepackage{xcolor}

\xdefinecolor{darc}{RGB}{6, 155, 212}
\colorlet{maincolor}{darc}

\setcapindent*{0em}

% do not sumber sections
\setcounter{secnumdepth}{0}

\setlength\intextsep{10pt plus 2pt minus 2pt}


% nicer lists
\usepackage{paralist}

% header and footer
\usepackage{scrpage2}
\pagestyle{scrheadings}
\clearscrheadfoot % remove all defaults
\automark{section} % \automark[<left side>]{<right side>}
%\ihead[\leftmark \\]{\leftmark \\ \rightmark}
\ihead{DARC OV Uetersen M21 \textcolor{maincolor}{\bfseries{}|} \textbf{Morsecode}}
\ohead{\pagemark}
%% optional arguments of \ihead and \ohead in [] is used at pagestyle plain
%% (e.g. at starting new chapters.)
\setkomafont{pagehead}{\scriptsize}
\setkomafont{pagenumber}{\scriptsize}
\setheadsepline{1pt}[\color{maincolor}]

\usepackage{hyperref}
\hypersetup{breaklinks=true,
            pdfborder={0 0 0},
            ngerman,
            pdfhighlight={/N},
            pdfauthor={Deutscher Amateur Radio Club Ortsverband Uetersen M21},
            pdftitle={Morsecode}}

\usepackage{tikz}
\usetikzlibrary{automata,positioning}
\newcommand{\basicdistance}{3ex}
\tikzset{
  every state/.style={
    inner sep=0pt,
    minimum width=2.3ex,
    minimum height=0pt,
    draw=maincolor,
    solid,
    thick,
    fill=maincolor!18
  },
  empty/.style={
    state,
    fill=black!9,
    draw=black!50
  },
  every initial by arrow/.style={
    thick
  },
  grow=right,
  level 1/.style={
    sibling distance=16*\basicdistance
  },
  level 2/.style={
    sibling distance=8*\basicdistance
  },
  level 3/.style={
    sibling distance=4*\basicdistance
  },
  level 4/.style={
    sibling distance=2*\basicdistance
  },
  level 5/.style={
    sibling distance=\basicdistance
  },
  level distance=3em
}

\newcommand{\dit}{{\LARGE\hskip1pt$\cdot$\hskip1pt}}
\newcommand{\dah}{{\LARGE\hskip1pt--\hskip1pt}}

\begin{document}
  \thispagestyle{plain}
  \begin{center}
    \large
  
    \pgfimage[width=4cm]{darc-logo.pdf}\\
    Ortsverand Uetersen M21
    
    \vskip1em
    
    \fontsize{36pt}{36pt}
    \selectfont
    \bfseries\color{maincolor} Morsecode
  \end{center}

  \section{Geschichte}
    Samuel Morse testete 1837 die erste Morsetaste. Dabei verwendete er
    einen Code der nur die zehn Ziffern kodieren konnte. Der erste Code, 
    mit dem auch Buchstabend kodiert werden können, wurde ein Jahr später
    von Alfred Lewis Vail, einem Mitarbeiter Morses, entwickelt.
    Dieser Code wurde als Morse Landline Code oder
    American Morse Code bei amerikanischen Eisenbahnen und den
    Telegrafenunternehmen bis in die 1960er Jahre eingesetzt.
    
    Der heute verwendete Code geht auf Friedrich Clemens Gerke zurück, der 1848
    zur Inbetriebnahme der elektromagnetischen Telegrafenverbindung zwischen
    Hamburg und Cuxhaven Vails Code vereinfachte. Dieser Code wurde nach
    einigen weiteren kleinen Änderungen 1865 auf dem Internationalen
    Telegraphenkongress in Paris standardisiert und später mit
    der Einführung der drahtlosen Telegrafie als Internationaler Morsecode
    von der Internationalen Fernmeldeunion (ITU) genormt.

  \section{Striche und Punkte}
    Mit dem Morsecode, manchmal auch Morsealphabet genannt, werden
    Buchstabend und Ziffern kodiert, sodass sie durch Ein- und Ausschalten
    eines Signals übertragen werden können.
    Der Morsecode besteht aus den Symbolen
    \begin{compactitem}[\color{maincolor}--]
      \item Strich (\dah, gesprochen Dah),
      \item Punkt (\dit, gesprochen Dit) und
      \item unterschiedlich langen Pausen.
    \end{compactitem}  
    Dabei gilt
    \begin{compactitem}[\color{maincolor}--]
      \item Die Länge eines Punktes ist die grundlegende Zeiteinheit.
      \item Ein Strich ist so lang wie drei Punkte.
      \item Eine Pause zwischen zwei Symbolen ist so lang wie ein Punkt.
      \item Eine Pause zwischen zwei Buchstaben des Klartextes ist so lang wie
        ein Strich.
      \item Eine Pause zwischen zwei Worten des Klartextes ist so lang wie
        sieben Punkte.
    \end{compactitem}
    
  \newpage
  
  \section{Standard-Codetabelle}
    \begin{minipage}[t]{.25\textwidth}
      \subsection{Buchstaben}
      
      \begin{tabular}{ll}
        A & \dit \dah \\
        B & \dah \dit \dit \dit \\
        C &	\dah \dit \dah \dit \\
        D &	\dah \dit \dit \\
        E &	\dit \\
        F &	\dit \dit \dah \dit \\
        G &	\dah \dah \dit \\
        H &	\dit \dit \dit \dit \\
        I &	\dit \dit \\
        J &	\dit \dah \dah \dah \\
        K &	\dah \dit \dah \\
        L &	\dit \dah \dit \dit \\
        M &	\dah \dah \\
        N &	\dah \dit \\
        O &	\dah \dah \dah \\
        P &	\dit \dah \dah \dit \\
        Q &	\dah \dah \dit \dah \\
        R &	\dit \dah \dit \\
        S &	\dit \dit \dit \\
        T &	\dah \\
        U &	\dit \dit \dah \\
        V &	\dit \dit \dit \dah \\
        W &	\dit \dah \dah \\
        X &	\dah \dit \dit \dah \\
        Y	& \dah \dit \dah \dah \\
        Z &	\dah \dah \dit \dit
      \end{tabular}
    \end{minipage}
    \qquad
    \begin{minipage}[t]{.25\textwidth}
      \subsection{Ziffern}

      \begin{tabular}{ll}
        0 &	\dah \dah \dah \dah \dah \\
        1	&	\dit \dah \dah \dah \dah \\
        2	&	\dit \dit \dah \dah \dah \\
        3	&	\dit \dit \dit \dah \dah \\
        4	&	\dit \dit \dit \dit \dah \\
        5	&	\dit \dit \dit \dit \dit \\
        6	&	\dah \dit \dit \dit \dit \\
        7	&	\dah \dah \dit \dit \dit \\
        8	&	\dah \dah \dah \dit \dit \\
        9	&	\dah \dah \dah \dah \dit
      \end{tabular}
    \end{minipage}
    \qquad
    \begin{minipage}[t]{.3\textwidth}
      \subsection{Satzzeichen}

      \begin{tabular}{lll}
        . & (AAA) &	\dit \dah \dit \dah \dit \dah \\
        , & (MIM) &	\dah \dah \dit \dit \dah \dah \\
        : & (OS)  &	\dah \dah \dah \dit \dit \dit \\
        ; & (NNN) &	\dah \dit \dah \dit \dah \dit \\
        ? & (IMI) &	\dit \dit \dah \dah \dit \dit \\
        = & &	\dah \dit \dit \dit \dah \\
        - & &	\dah \dit \dit \dit \dit \dah \\
        \_ & (UK) &	\dit \dit \dah \dah \dit \dah \\
        ( & (KN) &	\dah \dit \dah \dah \dit \\
        ) & (KK) &	\dah \dit \dah \dah \dit \dah \\
        + & (AR) &	\dit \dah \dit \dah \dit \\
        / & (DN) &	\dah \dit \dit \dah \dit \\
        @ & (AC) &	\dit \dah \dah \dit \dah \dit
      \end{tabular}
    \end{minipage}
    
    \vfill
    
    \hfill
    \begin{minipage}{8cm}
      \pgfimage[width=8cm]{key}
      \scriptsize\color{gray}
      \copyright\ Wikipedia-Benutzer Hgrobe/gallery/2008,
        Creative Commons Attribution 3.0 Unported
    \end{minipage}
    \hfill\strut
    
    \newpage
    
    \section{Dichotomischer Suchbaum}
    
    Mit dem folgenden Suchbaum kann Morsecode leicht dekodiert werden:
    Man beginnt am Startknoben auf der linken Seite
    und wechselt für jeden Strich
    zum oberen Kindknoten (gestrichelte Kante)
    und für jeden Punkt zum unteren Kindknoten (gepunktete Kante) bis zum
    Ende des Codewortes. 
    
    \begin{center}
      \begin{tikzpicture}[thick]
        \node[state] (start) {}
           child[dashed] { node[state] {T}
             child[dashed] { node[state] {M}
               child[dashed] { node[state] {G}
                 child[dashed] { node[state] {Q} }
                 child[dotted] { node[state] {Z}
                   child { node[coordinate] {} edge from parent[draw=none] }
                   child[dotted] { node[state] {7}
                     child { node[coordinate] {} edge from parent[draw=none] }
                     child[dotted] { node[state] {:} }
                   }
                 }
               }
               child[dotted] { node[state] {O}
                 child[dashed] { node[empty] {}
                   child[dashed] { node[state] {0} }
                   child[dotted] { node[state] {9} }
                 }
                 child[dotted] { node[empty] {}
                   child[dashed] { node[empty] {}
                     child[dashed] {node[state] {,}}
                     child { node[coordinate] {} edge from parent[draw=none] }
                   }
                   child[dotted] { node[state] {8} }
                 }
               }
             }
             child[dotted] { node[state] {N}
               child[dashed] { node[state] {K}
                 child[dashed] { node[state] {Y}
                   child { node[coordinate] {} edge from parent[draw=none] }
                   child[dotted] { node[state] {(}
                     child[dashed] { node[state] {)} }
                     child { node[coordinate] {} edge from parent[draw=none] }
                   }
                 }
                 child[dotted] { node[state] {C}
                   child[dashed] { node[empty] {}
                     child { node[coordinate] {} edge from parent[draw=none] }
                     child[dotted] { node[state] {;} }
                   }
                   child { node[coordinate] {} edge from parent[draw=none] }
                 }
               }
               child[dotted] { node[state] {D}
                 child[dashed] { node[state] {X}
                   child { node[coordinate] {} edge from parent[draw=none] }
                   child[dotted] { node[state] {/}
                     child[dashed] { node[state] {@} }
                     child { node[coordinate] {} edge from parent[draw=none] }
                   }
                 }
                 child[dotted] { node[state] {B}
                   child[dashed] { node[state] {=} }
                   child[dotted] { node[state] {6}
                     child[dashed] { node[state] {-} }
                     child { node[coordinate] {} edge from parent[draw=none] }
                   }
                 }
               }
             }
           }
           child[dotted] { node[state] {E}
             child[dashed] { node[state] {A}
               child[dashed] { node[state] {W}
                 child[dashed] { node[state] {J}
                   child[dashed] { node[state] {1} }
                   child { node[coordinate] {} edge from parent[draw=none] }
                 }
                 child[dotted] { node[state] {P} }
               }
               child[dotted] { node[state] {R}
                 child[dashed] { node[empty] {}
                   child { node[coordinate] {} edge from parent[draw=none] }
                   child[dotted] { node[state] {+}
                     child[dashed] { node[state] {.} }
                     child { node[coordinate] {} edge from parent[draw=none] }
                   }
                 }
                 child[dotted] { node[state] {L} }
               }
             }
             child[dotted] { node[state] {I}
               child[dashed] { node[state] {U}
                 child[dashed] { node[empty] {}
                   child[dashed] { node[state] {2} }
                   child[dotted] { node[empty] {}
                     child[dashed] { node[state] {\_} }
                     child[dotted] { node[state] {?} }
                   }
                 }
                 child[dotted] { node[state] {F} }
               }
               child[dotted] { node[state] {S}
                 child[dashed] { node[state] {V}
                   child[dashed] { node[state] {3} }
                   child { node[coordinate] {} edge from parent[draw=none] }
                 }
                 child[dotted] { node[state] {H}
                   child[dashed] { node[state] {4} }
                   child[dotted] { node[state] {5} }
                 }
               }
             }
           };
        \node[left=1.5em of start] (startlabel) {start};
        \draw
          (startlabel) edge[->] (start);
      \end{tikzpicture}
    \end{center}
    
    \newpage
    
    \section{Übertragungsrate}
    
    \section{Kontakt zum Ortsverband Uetersen M21}
\end{document}

